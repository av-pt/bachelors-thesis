\chapter{Conclusion}\label{conclusion}
% Short summary
\textcolor{violet}{
TBD: Summary, conclusion, outlook
}

Conclude from end of last section: (Maybe throw out direct speech), use supra-segmental features, use more character-based methods.

%Care has been taken to make all research easily reproducible and understandable (source-code READMES etc.).

% Discussion
% Initially suspected no improvement with transcription approaches. Probably too much information is lost in transcribing.
% Where to put this?


%Outlook:\\
%- Improve usability of unmasking, cite DH paper (Juola 2007), got a request for use of unmasking
%- Use large PAN20 dataset on better hardware.
%- Closer inspection of possible correlation: vocab size ~ evaluation results
%- Down-sample transcriptions so that vocab size is the same.
%- Feature set analysis, what martin said.
%- Use more clever methods that infer meaning by phonetic markers, e.g., intonation (questions with "?")
%- More thorough, e.g. Grid search, for best parameters


% "Clever Features"
% - n-gram approach instead of word-tokenization
% - phonetic richness / complexity (over time)
% - vowel-phoneme over time
% - sound-classes over time (e.g. hard sounds / plosives compared to soft sounds)
% - sound classes: consonants and place of articulation (e.g. labial, nasal, ...)