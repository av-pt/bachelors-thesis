\chapter{Introduction}\label{introduction}
With the widespread availability and use of text as a medium of information transfer, the task of identifying authorship of given texts has become one of the main focuses of stylometric analysis.
In this paper, we specifically tackle the problem of Authorship Verification which consists of classifying whether two given texts were written by the same author or not.
Underlying our approach is the idea that authors have a \textit{phonetic preference}, based on which they produce different texts.
\cite{ladefoged2014courseInPhonetics} titles this preference the "phonetics of the individual" and states that " [...] there can be little doubt that the set of phonetic habits and memories that each speaker possesses is different from those of every other speaker of the language".
By applying a range of phonetic transcription systems, we aim to emphasize the phonetically relevant features implanted into texts by their authors.
Then, we use the transcribed texts to train two well-known Authorship Verification classifiers.
We are aware that by assuming an authors' pronunciation to be, for example, a standard British one, some information is lost in the process of automatic transcription.
We will discuss the implications of this.
Nevertheless, we anticipate that the phonetic information encoded into plain text and boosted by transcribing gives the classifiers an advantage over more na\"{\i}ve methods.\\
In this paper, we will answer whether phonetic transcriptions are useful to boost certain features in the context of Authorship Verification.
Also, we will implement a new cross-validation functionality for both algorithms used.
All code and implementations will be open-sourced at \url{https://github.com/torond/bachelors-thesis} with an emphasis on ease of reproduction.









% % Showing natbib citation commands
% Let us get started by citing \citet{manning:1999}!
% So what did \citeauthor{manning:1999} do in % \citeyear{manning:1999}?
% Good question!
% % Showing hyperref reference commands
% Maybe it is answered in \autoref{introduction} on % page~\pageref{introduction}?
% Just to have something show up in the list of figures, I % included \autoref{fig:a}.
% \begin{figure}[bt]% bottom or top of page (for small % figures/tables)
%   \begin{center}{\huge\bf A}\end{center}
%   \caption{The first letter in the Roman % alphabet.}\label{fig:a}
% \end{figure}
% % \formatdate (or \formatdateshort)
% This date does not exist: \formatdateshort{30}{2}{2014}
% and is the same as \formatdate{30}{2}{2014}.
% % An example table
% And here is some table with some numbers % (\autoref{tab:numbers})
% which deserves to be on an extra page.
% \begin{table}[p]% extra page (usually for large figures/tables)
%   \caption{Tables have their captions above, figures below.}
%   \begin{center}
%     \begin{tabular}{lccc}\toprule
%       \multicolumn{4}{c}{Some numbers}\\\midrule
%       & 1999 & 2000 & 2001 \\\cmidrule(l){2-4}
%       % cmidrule: A line from 2nd to 4th column, trimmed on the % left hand side
%       Distance (km) & 23 & 18 & 42 \\
%       Awesomeness (aws) & 3.2 & 8.1 & 2.4 \\\bottomrule
%     \end{tabular}
%   \end{center}\label{tab:numbers}%
% \end{table}