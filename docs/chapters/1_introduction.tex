\chapter{Introduction}\label{introduction}
With the widespread availability and use of text as a medium of information transfer, the problem of identifying authorship of given texts has become one of the main focuses of stylometric analysis.
In this paper, we specifically tackle the task of Authorship Verification which consists of classifying whether two given texts were written by the same author or not.
Underlying our approach is the hypothesis that authors have a \textit{phonetic preference}, based on which they produce different texts.
\cite{ladefoged2014courseInPhonetics} titles this preference the ``phonetics of the individual'' and states that ``[\ldots] the set of phonetic habits and memories that each speaker possesses is different from those of every other speaker of the language''.
By applying phonetic transcription systems of varied granularity to the data used, we aim to emphasize these phonetically relevant features implanted into the texts by their authors.
Then, we use the transcribed data to train two well-known Authorship Verification classifiers.
By evaluating the results with standard measures used in Authorship Verification, we aim to answer the following questions:
\begin{itemize}
  \item Does the prior phonetic transcription of texts improve the performance of the algorithms over using verbatim text?
  \item Are the results correlated to the granularities of the transcriptions?
\end{itemize}
We transcribe textual data to North American English pronunciation.
We are aware that in this process some author-specific information is lost and will discuss the implications of this.
Nevertheless, we anticipate that the phonetic information encoded into plain text and boosted by transcribing gives the classifiers an advantage over more na\"{\i}ve methods.
All code attributions are open-sourced at \url{https://github.com/torond/bachelors-thesis} with an emphasis on ease of reproduction.