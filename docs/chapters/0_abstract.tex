\begin{abstract}
    \centerline{
\begin{minipage}{0.81\textwidth}
Authorship Verification is the task of deciding whether two texts were written by the same author or by different authors.
We hypothesize that authors have a \textit{phonetic preference}, based on which they produce texts, and that we can use this phonetic information to aid in classification.
Using a range of phonetic transcription systems of different granularity, we examine the viability of using transcription-based features in two well-known Authorship Verification algorithms.
We find that the use of phonetic representations of text does not yield an improvement in performance.
In fact, for many configurations we record statistically significant decreases in performance.
We propose three possible explanations for the negative results.
For reproducibility, all code is published as open-source.
\end{minipage}}
\end{abstract}